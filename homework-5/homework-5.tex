\documentclass[11pt, oneside]{article} 
\usepackage{geometry}               
\geometry{letterpaper}                  
\usepackage{graphicx}	
\usepackage{amssymb}
\usepackage{listings}
\usepackage{color}
\usepackage[T1]{fontenc}
\usepackage{titling}

\setlength{\droptitle}{-10em}

\definecolor{mygreen}{rgb}{0,0.6,0}
\definecolor{mygray}{rgb}{0.5,0.5,0.5}
\definecolor{mymauve}{rgb}{0.58,0,0.82}

\lstset{ %
  backgroundcolor=\color{white},   % choose the background color; you must add \usepackage{color} or \usepackage{xcolor}
  basicstyle=\footnotesize,        % the size of the fonts that are used for the code
  breakatwhitespace=false,         % sets if automatic breaks should only happen at whitespace
  breaklines=true,                 % sets automatic line breaking
  commentstyle=\color{mygreen},    % comment style
  deletekeywords={...},            % if you want to delete keywords from the given language
  escapeinside={\%*}{*)},          % if you want to add LaTeX within your code
  extendedchars=true,              % lets you use non-ASCII characters; for 8-bits encodings only, does not work with UTF-8
  frame=single,                    % adds a frame around the code
  keepspaces=true,                 % keeps spaces in text, useful for keeping indentation of code (possibly needs columns=flexible)
  keywordstyle=\color{blue},       % keyword style
  numbers=left,                    % where to put the line-numbers; possible values are (none, left, right)
  numbersep=5pt,                   % how far the line-numbers are from the code
  numberstyle=\tiny\color{mygray}, % the style that is used for the line-numbers
  rulecolor=\color{black},         % if not set, the frame-color may be changed on line-breaks within not-black text (e.g. comments (green here))
  stepnumber=2,                    % the step between two line-numbers. If it's 1, each line will be numbered
  stringstyle=\color{mymauve},     % string literal style
  tabsize=2,                       % sets default tabsize to 2 spaces
}

\title{Homework 5}
\author{Andrew Kowalczyk}
\date{\today}							% Activate to display a given date or no date

\begin{document}
\maketitle

\section*{3}
For C and Go, go to the links provided to see the function in action. Codepad for C and play.golang.org for Go.

\subsection*{Python}
\lstinputlisting[language=Python]{code/list_min.py}

\subsection*{C}
\lstinputlisting[language=C]{code/arraymin.c}

\subsection*{Javascript}
\lstinputlisting{code/arrayMin.js}

\subsection*{Go}
\lstinputlisting{code/ArrayMin.go}

\section*{4}
There is a possibility of getting stuck in an infinite loop. The user would potentially need to stop the program on their own if this were to happen.

\section*{5}
\subsection*{Javascript}
\lstinputlisting{code/subroutineOrder.js}

We can see after running this code in a shell or jsFiddle that JavaScript evaluate subroutines in the order that they are passed into a function (left-to-right).

\section*{6}
If the program outputs $0, 1, 2, 3, 4, 5, 6, 7, 8, 9$ this is how the program runs: It first pushes the return address of i on the stack (which is 0). Print i out. The increment it by 1. When foo is called a second time, it looks up the variable i again and simply increments it. Basically, the i overlayed the j in memory. If the program ran in this way, it is likely that the stack was cleared when the program was run.

There is also a possibility that one can see all zeroes. This is because when the program allocated a stack frame for the i, it allocated it in such a place where it just so happened to be 0. This could mean that the stack was not initialized on that particular system.

\section*{8}
The old version of Fortran printed 3 because it passed by reference. The modern version of Fortran prints 2 because it passes pointers to copies of rvalues. This is due to the fact that the compiler put the value of 2 (literal) in memory when foo was first called. Whenever there is 2 in the program, the compiler told it to look in that memory address. The value was changed when foo was called thus explaining why it printed 3.

\section*{10}
\subsection*{Call by value: $1, [2, 3, 4]$ is printed.}
\subsection*{Call by value-result: $2, [2, 2, 4]$ is printed.}
\subsection*{Call by reference: $2, [2, 3, 4]$ is printed.}
\subsection*{Call by name: $2, [2, 3, 4]$ is printed.}

\section*{11}
\lstinputlisting[language=Java]{code/queue.js}

\section*{12 - XC}
This is a bad idea because an object should not (cannot) have more than one class. As opposed to inheritance (\textit{i.e. IS-A}), this society of classes should be built by aggregation (\textit{i.e. HAS-A}). Aggregation should not be confused with composition. In other words, each $Person$ HAS-A $Job$. The class $Job$ can live it on its own without a $Person$ having that particular $Job$. Each $Job$ will have its own class to store properties specific to that $Job$. The person class should be the only class denoting a person. This person class can have its set of jobs or roles as a property.

\section*{13}

\subsection*{Java}
\lstinputlisting[language=Java]{code/OddGenerator.java}
\subsection*{Python}
\lstinputlisting[language=Python]{code/odd_generator.py}
\subsection*{Javascript}
\lstinputlisting{code/nextOdd.js}
\subsection*{C++}
\lstinputlisting[language=C++]{code/oddGenerator.cpp}

\end{document}  